\section{Motivation}
Calculating the spectral density of a matrix is trivial, when its eigenvalues are already known.
However, this is mostly not the case and calculating eigenvalues of very large matrices is time- and energy intensive.
At the same time, the DOS as a kind of probability density over the distribution of eigenvalues is of great interest in many fields.
Thus, there is a need for methods which approximate the spectral density at low cost.
The problem with this is that $\phi(t)$ the Delta distribution is not a \emph{function} as we know it,
that can be evaluated at each point.\\
A more intuitive idea would be to choose an intervall $I \in \R$ such that the spectrum of $A$, $\sigma(A)$, is a subset of $I$.
Now choose $k$ points $t_i$ in $I$, such that the intervall is divided in sub intervals:
$$\{t_i\}_{i = 1}^k \subset I \quad \text{mit} \quad \bigcup_{i = 1}^{k - 1} [t_i, t_{i+1}] = I$$
Now count the eigenvalues in every sub interval.
Then calculate the average value of $\phi(t)$ in every intervall with $\nu_{[a, b]}$ from equation \ref{eq:nu_a_b}.
The results is histograms, which with increasingly smaller subintervalls, that is to say bigger $k$ and $(t_{i+1} - t_i) \longrightarrow 0$, approach the spectral density.\\
To count the eigenvalues in the intervals, there is means like for example the Sylvestreschen Trägheitssatz.
The details of this method are not part of this work,
it would be necessary to calculate a decomposition of $A - t_i I = LDL^T$ for all $t_i$ \cite{golubvanloan}.
We prefer a method in which $A$ is multiplied with vectors, which is in bigger dimensions.\\
For simplicity we are going to assume, that $A$ is szmmetric and real.
The extension to hermetian matrices is simple in comparison.