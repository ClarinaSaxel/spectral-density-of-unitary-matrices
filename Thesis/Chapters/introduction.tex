We begin this thesis by examining unitary matrices and their fundamental properties.
Then we revisit the Cayley transform to finally close in on the spectral density.
Upon this we should be well-equipped to proceed with the investigation afterwards.

\section{Unitary Matrices}

We first recall two definitions for important real matrices that we then extend to complex matrices.
The index $^T$ marks the transpose of a matrix.
As common in literature, $I_n$ denotes the identity matrix of size $n$.

\begin{definition}[Symmetric matrix]
    Let $A$ be a real, square matrix of size $n$.
    Then $A$ is called \emph{symmetric} if $A^T = A$.
\end{definition}

The following definition is for matrices with their transpose as their inverse.

\begin{definition}[Orthogonal matrix]
    Let $A$ be a real, square matrix of size $n$.
    Then $A$ is called \emph{orthogonal} if $A^T \cdot A = I_n$.
\end{definition}

The complex equivalent of a real symmetric matrix is a \emph{Hermitian matrix}.
Note that $A^*$ is \emph{conjugate transpose} of the matrix $A$ with all of its entries complex conjugated and transposed.

\begin{definition}[Hermitian matrix]
    Let $A$ be a complex square matrix of size $n$.
    Then $A$ is called \emph{Hermitian} if $A^* = A$.
\end{definition}

Throughout this thesis, $A$ will denote a complex, square matrix of size $n$,
unless stated otherwise.
To reference a Hermitian matrix, we will use the letter $H$.

We now examine the eigenvalues of Hermitian matrices.

Let $H = H^*$ and $H v = \lambda v$ for a complex vector $v \neq 0$ of size $n$ and a scalar $\lambda \in \C$.
Consider now the inner product $ v^* v$.

\begin{equation} \label{eq:real_symmetric_inner_product}
    \lambda v^* v = v^* \left( \lambda v \right)
    = v^* \left( H v \right)
    = \left(v^* H \right) v
    = \left( H^* v \right)^* v
    = \left( H v \right)^* v
    = (\lambda v)^* v
    = \overline{\lambda} v^* v
\end{equation}

Since we have that $v \neq 0$ it follows that $v^* v \neq 0$
and therefore $\lambda = \overline{\lambda}$, that is to say $\lambda$ is real.
This means that all eigenvalues of Hermitian matrices are real numbers.
It follows that all eigenvalues of symmetric matrices are also real numbers,
since they are a special case of Hermitian matrices.
Now we can define the complex equivalent of orthogonal matrices:

\begin{definition}[Unitary matrix]
    A matrix $A$ is called \emph{unitary} if $A^* \cdot A = I_n$.
\end{definition}

We will oftentimes denote unitary matrices by using $U$ as a reference.
It is easy to see that orthogonal matrices are a special case of unitary matrices,
since $A^T = A^*$ for all real matrices.\\
Consider a unitary matrix $U$ and an eigenpair $(\lambda, v)$ of $U$.
The complex conjugate of the eigenvalue equation $U v = \lambda v$ is

\begin{equation} \label{eq:eigenvalue_equation_complex_conjugate}
    v^* U^* = v^* \overline{\lambda} = \overline{\lambda} v^*
\end{equation}

We calculate

\begin{align*}
    v^* v = v^* U^* U v = v^* \overline{\lambda} \lambda v = \overline{\lambda} \lambda v^* v = \left| \lambda \right|^2 v^* v
\end{align*}

Similarly to above, we can divide by $v^* v$ to obtain

\begin{equation} \label{eq:unitary_eigenvalues}
    1 = \left| \lambda \right|^2 = \left| \lambda \right|
\end{equation}

meaning that all eigenvalues of unitary matrices have a length of $1$ and are thus situated on the unit circle.
As they are a special case of unitary matrices, the same goes for orthogonal matrices.
This property is crucial, as it enables the application of the Cayley transform,
introduced in the following section.

There is a special group of matrices that all the matrices we have defined so far belong to.

\begin{definition}[Normal matrix]
    A matrix $A$ is called \emph{normal} if it commutes with its conjugate transpose,
    that is to say $A^* A = A A^*$.
\end{definition}

It is easy to see that both Hermitian and unitary matrices are normal matrices
and that the notion includes real symmetric and orthogonal matrices as special cases.
The spectral theorem states that normal matrices can be diagonalized by a unitary matrix.
That means, for any normal matrix $A$, there exists a unitary matrix $U$ such that $A = U \Lambda U^*$,
where $\Lambda = \diag(\lambda_1, \ldots, \lambda_n)$ with $\lambda_1, \ldots, \lambda_n$ being the eigenvalues of $A$.
This is essential for applying a function to a matrix, which we will get to in the next section.

\section{Cayley Transform}

Before giving the central definition of this section,
we will first enxtend the concept of functions to normal matrices,
allowing us to map one matrix to another.

\begin{definition}[Matrix function on normal matrices]
    Let $A$ be a normal matrix and let $f: \C \to \C$ be a function that is defined on the spectrum of $A$,
    $\sigma(A) = \{\lambda_1, \ldots, \lambda_n\}$.
    Then the \emph{matrix function} $f(A)$ is defined as
    \[
    f(A) := U f(\Lambda) U^* = U \diag(f(\lambda_1), \ldots, f(\lambda_n)) U^*
    \]
    where $U$ is the matrix of eigenvectors of $A$ and $\Lambda = \diag(\lambda_1, \ldots, \lambda_n)$ is the diagonal matrix of eigenvalues.
\end{definition}

Now we can define the \emph{Cayley transform}, which is a specific matrix function, 
that establishes a correspondence between Hermitian and unitary matrices,
allowing spectral properties to be translated between these two important classes.

For a complex number $z \in \mathbb{C}$ with $z \neq -i$, the Cayley transform is defined as
\[
\varphi(z) = \frac{i - z}{i + z}.
\]

This function maps the real line to the unit circle in the complex plane.

\vspace{0.5cm}

\begin{figure}[ht]
    \centering
    \begin{tikzpicture}

        % The real line
        \begin{axis}[
            axis lines = middle,
            name = realline,
            height = 4cm, width = 4cm,
            major tick length = 2ex,
            scale only axis,
            xlabel={$\re$},
            xlabel style = {
                at={(1.075,0.435)},
                anchor=south
            },
            ylabel={$\im$},
            ylabel style = {
                at={(0.415,1.07)},
                anchor=west
            },
            xmin = -2.5, xmax = 2.5,
            xtick = {-2,-1,1,2},
            ymin = -1.5, ymax = 1.5,
            ytick = {-1,1},
            yticklabels={$-i$,$i$}
        ]

            % (-\inf, -1)
            \addplot[
                domain = -3:-1,
                color = orange,
                style = thick
            ]
            ({x},{0});

            % [-1, 1]
            \addplot[
                domain = -1:1,
                color = blue,
                style = thick
            ]
            ({x},{0});

            % (1, \inf)
            \addplot[
                domain = 1:2.4,
                color = red,
                style = thick
            ]
            ({x},{0});

        \end{axis}

        % Cayley Transform
        \begin{axis}[
            at={(realline.east)},
            xshift=2.5cm,
            anchor=west,
            axis lines = middle,
            name = cayleytransform,
            height = 4cm, width = 4cm,
            scale only axis,
            xlabel={$\re$},
            xlabel style = {
                at={(1.075,0.435)},
                anchor=south
            },
            ylabel={$\im$},
            ylabel style = {
                at={(0.415,1.07)},
                anchor=west
            },
            ticks=none,
            xmin = -1.5, xmax = 1.5,
            ymin = -1.5, ymax = 1.5
        ]
            
            % Unit circle
            \addplot[
                domain = 0 : 2*pi,
                samples = 100
            ]
            ({cos(deg(x))}, {sin(deg(x))});

            % Image of (-\inf, -1)
            \addplot[
                domain = -200 : -1,
                color = orange,
                style = thick,
                samples = 1150
            ]
            ({(1 - x*x)/(1 + x*x)}, {(2*x)/(1 + x*x)});

            % Image of [-1, 1]
            \addplot[
                domain = -1 : 1,
                color = blue,
                style = thick,
                samples = 100
            ]
            ({(1 - x*x)/(1 + x*x)}, {(2*x)/(1 + x*x)});

            % Image of (1, \inf)
            \addplot[
                domain = 1 : 200,
                color = red,
                style = thick,
                samples = 1150
            ]
            ({(1 - x*x)/(1 + x*x)}, {(2*x)/(1 + x*x)});

        \end{axis}

        \path[->,out=45,in=135] 
            ($(realline.east)+(-0.5cm,1cm)$) edge
            node[auto,above] {$\varphi(z)$}
            ($(cayleytransform.west)+(0.5cm,1cm)$);

    \end{tikzpicture}
\end{figure}

% The real and imaginary values for the plots are calculated as follows:
%
%
% i - x    (i - x) * (i - x)     -1 - 2ix + x*x     -1 + x*x         -2x        1 - x*x         2x
% ----- = ------------------- = ---------------- = ---------- + i ---------- = --------- + i ---------
% i + x    (i + x) * (i - x)        -1 - x*x        -1 - x*x       -1 - x*x     1 + x*x       1 + x*x

For matrices, the Cayley transform maps a Hermitian matrix $H$ (with $i + H$ invertible) to a unitary matrix $U$ via
\[
U = (i - H)(i + H)^{-1}.
\]
The condition of $i + H$ being invertible is the same as requiring that $H$ does not have $-i$ as an eigenvalue.
If (and only if) that were the case, then we had $H v = -i v$ for some eigenvector $v \neq \mathbf{0}$,
and therefore 
\[
(i + H) v = i v + H v = i v + (-i v) = 0.
\]

Since Hermitian matrices have only real eigenvalues as discussed above, $-i$ can never be an eigenvalue.
Conversely, given a unitary matrix $U$ (with $U \neq -I_n$),
the inverse Cayley transform yields a Hermitian matrix:
\[
H = i (I_n - U)(I_n + U)^{-1}.
\]

This will be relevant later, as we can then use $\varphi$ to transform unitary matrices into symmetric ones and vice versa.
In particular, it enables the transfer of spectral density results,
which will be explored in the following sections.

\section{Spectral Density}

To get to the notion of the spectral density,
we will first need some more basic definitions to build on.

\begin{definition}[Linear functional]
    Let $V$ be a vectorspace over $\C$.
    A \emph{linear functional} $T$ is a linear map $T: V \to \C$.
\end{definition}

A simple example for such a functional is evaluation at a point:

\begin{equation} \label{eq:functional_example_simple}
    T: \Cinfty(\C) \to \C, \qquad f \mapsto f(0)
\end{equation}

where $\Cinfty(\C)$ denotes the space of infinitely differentiable (smooth) functions over $\C$.

Another example is integration against a fixed complex-valued function $g$:

\begin{equation} \label{eq:functional_example_integral}
    T_g: \Cinfty(\C) \to \C, \qquad f \mapsto \int\limits_{\C} g \cdot f \dx
\end{equation}

Now, let $\Omega \subset \C^n$ be open and non-empty.
A \emph{test function} is a smooth function with compact support defined on $\Omega$.
The space of all test functions on $\Omega$ is usually denoted by $\mathcal{E}$.
\emph{Distributions} generalize the concept of functions
and allow for differentiation in cases where classical differentiation is not possible.

\begin{definition}[Distribution] \label{def:Distribution}
    Let $\Omega \subset \C^n$ be open,
    and let $\mathcal{E}$ be the space of complex-valued test functions over $\Omega$.
    A \emph{distribution} $T$ is a continuous linear functional $T: \mathcal{E} \to \C$.
    That is, for all $g_1, g_2 \in \mathcal{E}$ and $\lambda \in \C$,
    \[
    T(g_1 + \lambda g_2) = T(g_1) + \lambda T(g_2)
    \]
    and for any sequence $\{g_n\}_{n \in \N}$ in $\mathcal{E}$ with $g_n \to g$,
    \[
    T(g_n) \to T(g).
    \]
\end{definition}

An important example of a distribution is the \emph{Dirac delta distribution}, which is defined as follows:

\begin{definition}[Dirac delta distribution]
    Let $\mathcal{E} = \Cinfty(\Omega)$ with $0 \in \Omega \subset \R^n$.
    Then
    $$\delta: \mathcal{E} \to \R, \quad f \mapsto f(0) \quad \text{with} \quad \delta(f) = \langle \delta, f \rangle = f(0)$$
\end{definition}

this distribution is often mistakenly referred to as a function,
although it is not a function in the classical sense.\\
The Dirac delta is characterized by the following property:

\[
\int\limits_{-\infty}^{\infty} f(x) \delta(x-a) \dx = \int\limits_{-\infty}^{\infty} f(x) \delta(a-x) \dx = f(a) \implies \int\limits_{-\infty}^{\infty} \delta(x-a) \dx = 1.
\]

This means that the Dirac delta distribution is zero everywhere except at the point $a$,
where it is infinitely high, such that the integral over it equals $1$.
We now have all the tools we need to define the central concept of this thesis:

\begin{definition}[Spectral density]
    Let $H$ be hermitian and sparse.
    The \emph{spectral density} is then defined as
    \[
    \phi(t) = \frac{1}{n} \sum_{j=1}^{n} \delta(t - \lambda_j)
    \]
    where $\delta$ is the Dirac delta distribution
    and $\lambda_j$ are the eigenvalues of $H$ in non-descending order.
\end{definition}

The number of eigenvalues in an interval $[a, b]$ can then be counted in the following manner:

\begin{equation} \label{eq:nu_a_b}
    \nu_{[a, b]} = \int\limits_a^b \sum_j \delta(t - \lambda_j) \dt \equiv \int\limits_a^b n \phi(t) \dt
\end{equation}

Random matrix theory tells us that the eigenvalues of random unitary matrices are distributed uniformly on the unit circle.
Thus, we would expect the spectral density to be uniformly distributed on the unit circle as well.
This is important as we think of the choice of random unitary matrices.

If we generate a random complex square matrix $A$,
there are multiple ways to obtain a unitary matrix $U$.
We will now compare the svd with the qr decomposition.

Before we can proceed to the motice of this thesis, we will need to define the following space:

\begin{definition}[Schwartz space over $\R$] \label{def:Schwartz space}
    The \emph{Schwartz space} over $\R$ consists of all smooth functions $f$ that decay rapidly to zero as $|x|$ approaches infinity \cite{richtmyer}.
    Formally,
    \[
    \SR := \left\{f \in \Cinfty(\R) \mid \forall p, k \in \N_0: \sup_{x \in \R} \left| x^p f^{(k)}(x) \right| < \infty \right\}
    \]
\end{definition}