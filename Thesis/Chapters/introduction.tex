To commence this investigation we begin by examining unitary matrices and their fundamental properties.
Then we revisit the Cayley transform to finally close in on the spectral density.
Upon this we should be well-equipped to proceed with the investigation afterwards.

\section{Unitary Matrices}

We first recall a definition that will be relevant in later chapters.
The index $^T$ marks the transpose of a matrix.
As common in a lot of literature, $I_n$ denotes the identity matrix of size $n$.

\begin{definition}[Orthogonal matrix]
    Let $A$ be a real, square matrix of size $n$.
    Then $A$ is called \emph{orthogonal} if $A^T \cdot A = I_n$.
\end{definition}

Throughout this thesis, $A$ will denote a complex, square matrix of size $n$, unless stated otherwise.
Note that $A^*$ is \emph{conjugate transpose} of the matrix $A$ with all of its entries complex conjugated and transposed.

\begin{definition}[Unitary matrix]
    A matrix $A$ is called \emph{unitary} if $A^* \cdot A = I_n$.
\end{definition}

We will oftentimes denote unitary matrices by using $U$ as a reference.
It is easy to see that orthogonal matrices are a special case of unitary matrices, since $A^T = A^*$ for all real matrices.\\
Consider a unitary matrix $U$; we are interested in determining its eigenvalues.
That means we have to solve the equation

\begin{equation} \label{eq:eigenvalue_equation}
    U \cdot v = \lambda \cdot v
\end{equation}

for a complex vector $v \neq \mathbf{0}$ of size $n$ and a scalar $\lambda \in \C$.
The complex conjugate of this equation is

\begin{equation} \label{eq:eigenvalue_equation_complex_conjugate}
    v^* \cdot U^* = v^* \cdot \lambda^* = \lambda^* \cdot v^*
\end{equation}

Put together, we calculate

\begin{align*}
    v^* \cdot v & = v^* \cdot U^* \cdot U \cdot v \\
    & = v^* \cdot \lambda^* \cdot \lambda \cdot v \\
    & = \lambda^* \cdot \lambda\cdot v^* \cdot v \\
    & = \left| \lambda \right|^2 \cdot v^* \cdot v
\end{align*}

Since we have that $v \neq \mathbf{0}$ it follows that $v^* \cdot v \neq 0$.
Therefore, we can divide by $v^* \cdot v$ to obtain

\begin{equation} \label{eq:unitary_eigenvalues}
    1 = \left| \lambda \right|^2 = \left| \lambda \right|
\end{equation}

meaning that all eigenvalues of unitary matrices have a length of $1$ and are thus situated on the unit circle.
For orthogonal matrices, this means their eigenvalues are either $1$ or $-1$.
This is an important property, as it justifies that we can make use of the Cayley transfrom introduced in the following section.

\section{Cayley Transform}

The Cayley transform or Cayley transformation is given by the simple function

$$\varphi(z) = (i-z)(i+z)^{-1}$$

This function maps the real line to the unit circle, and more specifically, the interval $[-1, 1]$ to the semi circle
$\{z = e^{i\theta}: \theta \in [-\pi/2,\pi/2]\}$ as shown below

\vspace{0.5cm}

\begin{figure}[ht]
    \centering
    \begin{tikzpicture}

        % The real line
        \begin{axis}[
            axis lines = middle,
            name = realline,
            height = 4cm, width = 4cm,
            major tick length = 2ex,
            scale only axis,
            xlabel={$\re$},
            xlabel style = {
                at={(1.075,0.435)},
                anchor=south
            },
            ylabel={$\im$},
            ylabel style = {
                at={(0.415,1.07)},
                anchor=west
            },
            xmin = -2.5, xmax = 2.5,
            xtick = {-2,-1,1,2},
            ymin = -1.5, ymax = 1.5,
            ytick = {-1,1},
            yticklabels={$-i$,$i$}
        ]

            % (-\inf, -1)
            \addplot[
                domain = -3:-1,
                color = orange,
                style = thick
            ]
            ({x},{0});

            % [-1, 1]
            \addplot[
                domain = -1:1,
                color = blue,
                style = thick
            ]
            ({x},{0});

            % (1, \inf)
            \addplot[
                domain = 1:2.4,
                color = red,
                style = thick
            ]
            ({x},{0});

        \end{axis}

        % Cayley Transform
        \begin{axis}[
            at={(realline.east)},
            xshift=2.5cm,
            anchor=west,
            axis lines = middle,
            name = cayleytransform,
            height = 4cm, width = 4cm,
            scale only axis,
            xlabel={$\re$},
            xlabel style = {
                at={(1.075,0.435)},
                anchor=south
            },
            ylabel={$\im$},
            ylabel style = {
                at={(0.415,1.07)},
                anchor=west
            },
            ticks=none,
            xmin = -1.5, xmax = 1.5,
            ymin = -1.5, ymax = 1.5
        ]
            
            % Unit circle
            \addplot[
                domain = 0 : 2*pi,
                samples = 100
            ]
            ({cos(deg(x))}, {sin(deg(x))});

            % Image of (-\inf, -1)
            \addplot[
                domain = -200 : -1,
                color = orange,
                style = thick,
                samples = 1150
            ]
            ({(1 - x*x)/(1 + x*x)}, {(2*x)/(1 + x*x)});

            % Image of [-1, 1]
            \addplot[
                domain = -1 : 1,
                color = blue,
                style = thick,
                samples = 100
            ]
            ({(1 - x*x)/(1 + x*x)}, {(2*x)/(1 + x*x)});

            % Image of (1, \inf)
            \addplot[
                domain = 1 : 200,
                color = red,
                style = thick,
                samples = 1150
            ]
            ({(1 - x*x)/(1 + x*x)}, {(2*x)/(1 + x*x)});

        \end{axis}

        \path[->,out=45,in=135] 
            ($(realline.east)+(-0.5cm,1cm)$) edge
            node[auto,above] {$\varphi(z)$}
            ($(cayleytransform.west)+(0.5cm,1cm)$);

    \end{tikzpicture}
\end{figure}

% The real and imaginary values for the plots are calculated as follows:
%
%
% i - x    (i - x) * (i - x)     -1 - 2ix + x*x     -1 + x*x         -2x        1 - x*x         2x
% ----- = ------------------- = ---------------- = ---------- + i ---------- = --------- + i ---------
% i + x    (i + x) * (i - x)        -1 - x*x        -1 - x*x       -1 - x*x     1 + x*x       1 + x*x

This will be relevant later, as we can then use $\varphi$ to transform unitary matrices into symmetric once and vice versa.
Now, let's look how the spectral density is defined in the next section.

\section{Spectral Density}

To get to the notion of the spectral density, we will first need some more basic definitions to build on.

\begin{definition}[linear functional]
    Let $V$ be a vectorspace over a field $\K$. A \emph{linear functional} $T$ is a linear function $T: V \to \K$.
    The space over all linear functionals $V \mapsto \K$ is called the \emph{dual space} $V'$.
\end{definition}

A simple example for such a functional would be

\begin{equation} \label{eq:functional_example_simple}
    T: \Cinfty(\R) \to \R, \qquad f \mapsto f(0)
\end{equation}

A more special case is the integral.

\begin{equation} \label{eq:functional_example_integral}
    T_g: \Cinfty(\C) \to \C, \qquad f \mapsto \int\limits_{\C} g \cdot f \dx
\end{equation}

This definition leads to the concept of a distribution which was introduced to get a method to differentiate where differentiation in the classical sense is not possible.

\begin{definition}[distribution]
    Let $\emptyset \neq \Omega \subset \R^n$ be open.
    Let $\mathcal{E}$ be the space of \emph{test functions} over $\Omega$.
    A \emph{distribution} $T$ is a function $T: \mathcal{E} \to \C$ where for all
    $g, g_1, g_2, \{g_n\}_{n \in \N} \in \mathcal{E}$
    with $\lim\limits_{n \to \infty} g_n \to g$ it holds:
    $$T(g_1 + \lambda g_2) = T(g_1) + \lambda T(g_2) \quad \text{und}\quad \lim\limits_{n \to \infty} T(g_n) \to T(g)$$
\end{definition}

So, put shortly: A distribution $T$ is a continuous and linear functional on $\mathcal{E}$.
Now the path is clear to define the so called Dirac delta distribution

\begin{definition}[Dirac delta function]
    Let $\mathcal{E} = \Cinfty(\Omega)$ with $0 \in \Omega \subset \R^n$.
    Then
    $$\delta: \mathcal{E} \to \R, f \mapsto f(0) \quad \text{mit} \quad \delta(f) = \langle \delta, f \rangle = f(0)$$
\end{definition}

An important feature of this definition is:

$$\int\limits_{-\infty}^{\infty} f(x) \delta(x-a) \dx = \int\limits_{-\infty}^{\infty} f(x) \delta(a-x) \dx = f(a) \implies \int\limits_{-\infty}^{\infty} \delta(x-a) \dx = 1$$

This distribution is often misleadingly labled as a function inspite of being decidedly not that.\\
We can now finally define the spectral density.

\begin{definition} [spectral density]
    Let $A \in \R^{n \times n}$, $A^T = A$ and $A$ sparce.
    Then, the spectral density is defined as
    $$\phi(t) = \frac{1}{n} \sum_{j=1}^{n} \delta(t - \lambda_j)$$
    where $\delta$ is the delta distribution and $\lambda_j$ are the eigenvalues of $A$ in non-descending order.
\end{definition}

The number of eigenvalues in an intervall $[a, b]$ can then be expressed as follows:

\begin{equation} \label{eq:nu_a_b}
    \nu_{[a, b]} = \int\limits_a^b \sum_j \delta(t - \lambda_j) \dt \equiv \int\limits_a^b n \phi(t) \dt
\end{equation}

\begin{definition} [Schwartz-space over $\R$] \label{def:Schwartz space}
    The Schwartz-space over $\R$ contains all smooth functions $f$,
    which fall fast enough against $0$, when $|x|$ gets closer to $\infty$. \cite{richtmyer}
    In Formeln
    $$\SR(\R) := \left\{f \in \Cinfty(\R) \mid \forall p, k \in \N_0: \sup_{x \in \R} \left| x^pf^{(k)}(x)\right| < \infty \right\}$$
    Im Weiteren werde ich das Symbol $\SR$ als Abkürzung für $\SR(\R)$ benutzen,
    da sich diese Arbeit allein mit dem reellen Kontext befasst.
\end{definition}